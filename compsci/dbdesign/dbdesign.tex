\documentclass[12pt, letterpaper]{report}
\usepackage{graphicx}
\usepackage{float}


\newfloat{dbfigure}{H}{lodbf}[section]
\floatname{dbfigure}{Database Figure}

%\floatstyle{ruled}
%\newfloat{code}{H}{loc}[section]
%\floatname{code}{Source Code}



\newcommand{\loadfigure}[2]{\includegraphics[scale=#1]{images/#2}}

\newcommand{\captionsource}[2]{
	\centering
	\caption[{#1}]{
	#1
	\\\hspace{\linewidth}
	\normalfont\scriptsize \textbf{Source:} #2}
}

\newcommand{\listfigure}[4]{
\begin{dbfigure}[H]
	\centering
	\loadfigure{#1}{#2}
	\captionsource{#3}{#4}
\end{dbfigure}
}


\begin{document}

\part{Databases}

\section{Definition}

A database is a persistent collection of data, and is typically more than one single collection of data. Database Management Systems (\textbf{DBMS})are a central repository of shared data, and data is accessed and transferred via a common language.

\listfigure{0.35}{dbmodel_1.png}{Shared Information Model}{Bryn Jefferies, USyd INFO2820}

\listfigure{0.35}{dbmodel_2.png}{Relational Database Model}{Bryn Jefferies, USyd INFO2820}

While the database transfers data through a common language, the set-up of the database varies from system to system. Below is an example of the structure of a DBMS:

\listfigure{0.45}{dbmodel_3.png}{DBMS Structure}{Bryn Jefferies, USyd INFO2820}

The language that most systems communicate with DBMSs is \textbf{SQL}, which stands for \textbf{Structured Query Language}. 

\part{Relational Data Model}

\part{Relational Algebra}

\part{Datalog}

\part{Complex SQL}

\part{SQL Hierarchies}

\part{Recursive SQL}

\part{Triggers}

\part{Multi-valued Dependencies}

\part{Database Security and Integrity}

\part{Schema Refinement}

\part{Normal Forms}

\part{ACID Transactions}

\part{Database Application Development}

\part{Transaction Management}

\part{Indexing and Tuning}






\end{document}